\section{MyBatis 动态 SQL}

MyBatis3采用了功能强大的OGNL(Object-Graph Navigation Language)表达式语言消除了许多其他标签。以下是 MyBatis3 的动态 SQL 在 XML 中支持的几种标签: if, choose(when, otherwise), trim(where, set), foreach, bind。

\subsection{IF 用法}

if 标签通常用于 WHERE,UPDATE,INSERT 语句中用于判断字段。

\subsubsection*{WHERE 中的 IF}

如果我们有这样的需求: 实现一个用户管理高级查询功能,根据输入的条件去检索用户信息。
\begin{itemize}
    \item 当只输入用户名时,需要根据用户名进行模糊查询;
    \item 当只输入邮箱时,根据邮箱进行完全匹配;
    \item 当同时输入用户名和邮箱时,用这两个条件去查询匹配的用户。
\end{itemize}

如果仅使用 MySQL 的语法,无法在一句语句中完成三种条件下的需求,需要通过 Java 程序进行判断后选择合适的 SQL 语句执行。但 MyBatis 使用 IF 标签可以在一句语句中完成该任务。

\begin{xml}
<select id="selectByUser" resultType="learn.mybatis.simple.model.SysUser">
    SELECT id, user_name, user_password, user_email, user_info, head_img, create_time
    FROM sys_user
    WHERE 1 = 1
    <if test="userName != null and userName !=''">
        AND user_name like concat('%',#{userName},'%')
    </if>
    <if test="userEmail != null and userEmail != ''">
        AND user_email = #{userEmail}
    </if>
</select>
\end{xml}

test的属性值是一个符合OGNL要求的判断表达式,表达式的结果可以是true或false,除此之外所有的非0值都为true,只有0为false。建议只用true或false作为结果。

OGNL 的详细用法会在下文介绍,判断条件 property == null 对任何类型字段都适用, property == '' 仅适用于 String 类型的字段。当有多个判断条件时可以用 and 或 or 进行连接。

注意 1=1 的条件 可以避免 SQL 语法错误导致的异常,这种写法将在下文被取代。

\subsubsection*{UPDATE 中的 IF}

现在要实现这样一个需求:只更新有变化的字段。需要注意,更新的时候不能将原来有值但没有发生变化的字段更新为空或null。通过if标签可以实现这种动态列更新。

一般情况下,MyBatis中选择性更新的方法名会以Selective作为后缀。在UserMapper.xml中添加对应的SQL语句,代码如下:

\begin{xml}
<update id="updateByIdSelective">
    UPDATE sys_user
    SET
    <if test="userName != null and userName != ''">
        user_name = #{userName},
    </if>
    <if test="userPassword != null and userPassword != ''">
        user_password = #{userPassword},
    </if>
    <if test="userEmail != null and userEmail != ''">
        user_email = #{userEmail},
    </if>
    <if test="userInfo != null and userInfo != ''">
        user_info = #{userInfo},
    </if>
    <if test="headImg != null">
        head_img = #{headImg, jdbcType = BLOB},
    </if>
    <if test = "createTime != null">
        create_time = #{createTime jdbcType = TIMESTAMP},
    </if>
    id = #{id}
    WHERE id = #{id}
</update>
\end{xml}

只要注意一下书写顺序,符不符合 sql 语句规范即可。

\subsubsection*{INSERT 中的 IF}

在数据库表中插入数据的时候,如果某一列的参数值不为空,就使用传入的值,如果传入参数为空,就使用数据库中的默认值(通常是空),而不使用传入的空值。使用if就可以实现这种动态插入列的功能。

\begin{xml}
<insert id="insert2" useGeneratedKeys="true" keyProperty="id">
    INSERT INTO sys_user (user_name, user_password,
        <if test="userEmail != null and userEmail != ''">
            user_email,
        </if>
        user_info, head_img, create_time)
    VALUES (
        #{userName}, #{userPassword},
        <if test="userEmail != null and userEmail != ''">
            #{user_email},
        </if>
        #{userEmail}, #{headImg jdbcType=BLOB}, #{createTime jdbcType = TIMESTAMP}
    )
</insert>
\end{xml}



















\newpage