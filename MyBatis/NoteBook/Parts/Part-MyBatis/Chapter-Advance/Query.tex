\section{MyBatis 高级查询}

\subsection{高级结果映射}

\subsubsection{一对一映射}

一对一映射因为不需要考虑是否存在重复数据,因此使用起来很简单,而且可以直接使用MyBatis的自动映射。

\paragraph*{使用自动映射处理一对一关系}

使用自动映射就是通过别名让MyBatis自动将值匹配到对应的字段上,简单的别名映射如user\_name 对应 userName。

除此之外 MyBatis 还支持复杂的属性映射,可以多层嵌套,例如将role.role\_name映射到role.roleName上。MyBatis会先查找role属性,如果存在role属性就创建role 对象,然后在role 对象中继续查找roleName,将role\_name的值绑定到role对象的roleName属性上。

在 SysUser 中添加属性:
\begin{Java}
public class SysUser {
    private SysRole role;
    public SysRole getRole() {
        return role;
    }
    public void setRole(SysRole role) {
        this.role = role;
    }
    // 其他代码
}
\end{Java}

\begin{xml}
<select id="selectUserAndRoleById" resultType="learn.mybatis.simple.model.SysUser">
    SELECT u.id,
           u.user_name,
           u.user_password,
           u.user_email,
           u.user_info,
           u.head_img,
           u.create_time,
           r.id "role.id",
           r.role_name "role.roleName",
           r.enabled "role.enabled",
           r.create_by "role.createBy",
           r.create_time "role.createBy"
    FROM sys_user AS u,
         sys_role AS r,
         sys_user_role AS ur
             INNER JOIN ur on u.id = ur.user_id
             INNER JOIN r on ur.role_id = r.id
    WHERE u.id = #{id}
</select>
\end{xml}

注意上述方法中sys\_role查询列的别名都是“role.”前缀,通过这种方式将role的属性都映射到了SysUser的role属性上。通过SQL日志可以看到已经查询出的一条数据,MyBatis将这条数据映射到了两个类中,像这种通过一次查询将结果映射到不同对象的方式,称之为关联的嵌套结果映射。

\paragraph*{使用 resultMap 配置一对一映射}

resultMap 方式,除了写起来繁,没其他缺点:

\begin{xml}
<resultMap id="userRoleMap" type="learn.mybatis.simple.model.SysUser">
    <id property="id" column="id"/>
    <result property="userName" column="user_name"/>
    <result property="userPassword" column="user_password"/>
    ......
    <result property="role.id" column="role_id">
    <result property="role.roleName" column="role_name">
    <result property="role.enabled" column="enabled">
    ......
</resultMap>
\end{xml}

通过生成器可以生成一些基础代码后进行扩充可以方便一点。

\begin{xml}
<resultMap id="userRoleMap" type="learn.mybatis.simple.model.SysUser" extends="userMap">
    <result property="role.id" column="role_id"/>
    <result property="role.roleName" column="role_roleName"/>
    <result property="role.enabled" column="role_enabled"/>
    <result property="role.createBy" column="role_createBy"/>
    <result property="role.createTime" column="role_create_ime" jdbcType="TIMESTAMP"/>
</resultMap>
\end{xml}

\paragraph*{使用 resultMap 的 association 标签配置一对一映射}

在resultMap中,association标签用于和一个复杂的类型进行关联,即用于一对一的关联配置。

在上面配置的基础上,再做修改,改成association标签的配置方式,代码如下:

\begin{xml}
<resultMap id="userRoleMap" type="learn.mybatis.simple.model.SysUser" extends="userMap">
    <association property="role" columnPrefix="role_" javaType="learn.mybatis.simple.model.SysRole">
        <result property="id" column="id"/>
        <result property="roleName" column="name"/>
        <result property="enabled" column="enabled"/>
        <result property="createBy" column="createBy"/>
        <result property="createTime" column="create_img" jdbcType="TIMESTAMP"/>
    </association>
</resultMap>
\end{xml}

<association> 标签包含如下属性:
\begin{itemize}
    \item property: 对应实体类你中的属性名,必填。
    \item javaType: 属性对应 java 类型。
    \item resultMap: 使用现有的 resultMap。
    \item columnPrefix: 配置前缀,下面就不用写了。
\end{itemize}

\paragraph*{association 标签的嵌套查询}

除了前面3种通过复杂的SQL查询获取结果,还可以利用简单的SQL通过多次查询转换为我们需要的结果,这种方式与根据业务逻辑手动执行多次SQL的方式相像,最后会将结果组合成一个对象。

association 标签的嵌套查询常用的属性如下:
\begin{itemize}
    \item select: 另一个映射查询的 id, MyBatis 会额外执行这个查询获取嵌套对象的结果。
    \item column: 列名,将主查询中列的结果作为嵌套查询的参数,配置方式如: column = {prop = col1m prop2 = col2} prop1 和 prop2 将作为嵌套查询的参数。
    \item fetchType: 数据加载方式,可选值为 lazy 和 eager,分别为延迟加载和积极加载,这个配置会覆盖全局的lazyLoadingEnabled配置。
\end{itemize}

使用嵌套查询的方式配置一个和前面功能一样的方法,首先在UserMapper.xml中创建如下的resultMap。

\begin{xml}
<resultMap id="userRoleMapSelect" type="learn.mybatis.simple.model.SysUser" extends="userMap">
    <association property="role" column="{id = role_id}" select="learn.mybatis.simple.mapper.RoleMapper.selectRoleById"/>
</resultMap>
\end{xml}

然后创建对应的 select 方法:

\begin{xml}
<select id="selectUserAndRoleByIdSelect" resultMap="userRoleMapSelect">
    SELECT u.id,
           u.user_name,
           u.user_password,
           u.user_email,
           u.user_info,
           u.head_img,
           u.create_time,
           ur.role.id
    FROM sys_user AS u,
         sys_user_role AS ur
             INNER JOIN ur on u.id = ur.user_id
    WHERE u.id = #{id}
</select>
\end{xml}