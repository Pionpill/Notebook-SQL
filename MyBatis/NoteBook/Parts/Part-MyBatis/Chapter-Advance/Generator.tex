\section{MyBatis 代码生成器}

作为一个优秀的程序员,“懒”是很重要的优点。不仅要会写代码,还要会利用(或自己实现)工具生成代码。MyBatis 的开发团队提供了一个很强大的代码生成器——MyBatis Generator(MBG) 来帮助我们缩减 SQL 语句。

MBG 的版本和 MyBatis 的版本没有直接关系,不同的MBG版本包含的参数可能不一样。

\subsection{使用 MyBatis Generator}

\subsubsection{配置文件}

在 src/main/resources 中创建 generator-config.xml 内容如下:

\begin{xml}
<?xml version="1.0" encoding="UTF-8"?>
<!DOCTYPE generatorConfiguration
        PUBLIC "-//mybatis.org//DTD MyBatis Generator Configuration 1.0//EN"
        "http://mybatis.org/dtd/mybatis-generator-config_1_0.dtd">
<generatorConfiguration>
    <properties resource="sql.properties"/>
    <context id="MySqlContext" targetRuntime="MyBatis3Simple" defaultModelType="flat">
        <property name="beginningDelimiter" value="`"/>
        <property name="endingDelimiter" value="`"/>

        <commentGenerator>
            <property name="suppressDate" value="true"/>
            <property name="addRemarkComments" value="true"/>
        </commentGenerator>

        <jdbcConnection driverClass="${driver}" connectionURL="${url}" userId="${username}" password="${password}">
        </jdbcConnection>

        <javaModelGenerator targetPackage="test.model" targetProject="src\main\java">
            <property name="trimStrings" value="true"/>
        </javaModelGenerator>

        <sqlMapGenerator targetPackage="test.xml" targetProject="src\main\resources"/>

        <javaClientGenerator type="XMLMAPPER" targetPackage="test.dao" targetProject="src\main\java"/>

        <table tableName="%">
            <generatedKey column="id" sqlStatement="MySql"/>
        </table>
    </context>
</generatorConfiguration>
\end{xml}

先不管这东西是干嘛的,先运行起来看看怎么用。

\subsubsection{运行 MyBatis Generaotr}

MBG 提供了多种运行方式,各有优缺点,常用的有以下几种:
\begin{itemize}
    \item 使用 Java 编写运行代码。
    \item 从命令提示符运行。
    \item 使用 MavenPlugin 运行。
    \item 使用IDE插件运行。
\end{itemize}

这里只介绍第一种方法,其他三种自行查阅资料。

在运行前,需要将对应的 jar 包添加到项目中,可以直接下载 jar 包或者通过 maven 引入依赖。

\begin{xml}
<dependency>
    <groupId>org.mybatis.generator</groupId>
    <artifactId>mybatis-generator-core</artifactId>
    <version>1.3.3</version>
</dependency>
\end{xml}

然后创建对应的 Generator 类:

\begin{Java}
public class Generator {
    public static void main(String[] args) throws Exception {
        //MBG 执行过程中的警告信息
        List<String> warnings = new ArrayList<String>();
        //当生成的代码重复时,覆盖原代码
        boolean overwrite = true;
        //读取我们的 MBG 配置文件
        InputStream is = Generator.class.getResourceAsStream("/generator/generatorConfig.xml");
        ConfigurationParser cp = new ConfigurationParser(warnings);
        Configuration config = cp.parseConfiguration(is);
        is.close();

        DefaultShellCallback callback = new DefaultShellCallback(overwrite);
        //创建 MBG
        MyBatisGenerator myBatisGenerator = new MyBatisGenerator(config, callback, warnings);
        //执行生成代码
        myBatisGenerator.generate(null);
        //输出警告信息
        for(String warning : warnings){
            System.out.println(warning);
        }
    }
}
\end{Java}

使用 Java 编码方式运行的好处是,generatorConfig.xml 配置的一些特殊的类(如commentGenerator标签中type属性配置的MyCommentGenerator类)只要在当前项目中,或者在当前项目的classpath中,就可以直接使用。使用其他方式时都需要特别配置才能在MBG 执行过程中找到 MyCommentGenerator 类并实例化,否则都会由于找不到这个类而抛出异常。

使用Java编码不方便的地方在于,它和当前项目是绑定在一起的,在Maven多子模块的情况下,可能需要增加编写代码量和配置量,配置多个,管理不方便。但是综合来说,这种方式出现的问题最少,配置最容易,因此推荐使用。

\subsection{XML 配置详解}

配置文件必备的基本信息如下:
\begin{xml}
<?xml version="1.0" encoding="UTF-8"?>
<!DOCTYPE generatorConfiguration
        PUBLIC "-//mybatis.org//DTD MyBatis Generator Configuration 1.0//EN"
        "http://mybatis.org/dtd/mybatis-generator-config_1_0.dtd">
<generatorConfiguration>
...
</generatorConfiguration>
\end{xml}

其中,头文件确定了对应的 DTD 文件用来规范格式。 generatorConfiguration 是根节点,他下面有三个子标签: 分别是 properties、classPathEntry和context。这三个标签的顺序必须与列举的顺序一致。

properties 标签用于指定一个需要解析的外部属性文件,引入属性文件后,后面需要配置的 JDBC 信息可以通过 \$\{property\} 的形式引入。这个标签最多可以配置1个,也可以不配置。properties标签包含resource和url两个属性,只能使用其中一个属性来指定,同时出现则会报错。
\begin{itemize}
    \item resource: 指定 classpath 下的属性文件,类似com/myproject/generatorConfig.properties这样的属性值。
    \item url: 指定文件系统上的特定位置,例如 file:///C:/myfolder/generatorConfig.properties。
\end{itemize}

第二个是classPathEntry标签。这个标签可以配置多个,也可以不配置。 classPathEntry标签最常见的用法是通过属性location指定驱动的路径,代码如下。
\begin{xml}
<classPathEntry location="E:\mysql\mysql-connector-java-5.1.29.jar"/>
\end{xml}

第三个是context标签。这个标签是最重要的,该标签至少配置1个,可以配置多个。context标签用于指定生成一组对象的环境。例如指定要连接的数据库,要生成对象的类型和要处理的数据库中的表。运行MBG的时候还可以指定要运行的context。context标签只有一个必选属性id,用来唯一确定该标签,此外还有几个可选属性:

\begin{itemize}
    \item defaultModelType: 定义 MBG 如何生成实体类,有以下选项:
    \begin{itemize}
        \item defaultModelType: 和 hierarchical 类似,如果一个表的主键只有一个字段,那么不会为该字段生成单独的实体类,而是会将该字段合并到基本实体类中。
        \item flat: 该模型只为每张表生成一个实体类。这个实体类包含表中的所有字段。这种模型最简单,推荐使用。
        \item hierarchical:如果表有主键,那么该模型会产生一个单独的主键实体类,如果表还有BLOB字段,则会为表生成一个包含所有BLOB字段的单独的实体类,然后为所有其他的字段另外生成一个单独的实体类。MBG会在所有生成的实体类之间维护一个继承关系。
    \end{itemize}
    \item  targetRuntime:此属性用于指定生成的代码的运行时环境,有以下选项:
    \begin{itemize}
        \item MyBatis3: 默认值。
        \item MyBatis3Simple: 这种情况不会生成与Example相关的方法。
    \end{itemize}
    \item introspectedColumnImpl: 该参数可以指定扩展 Introspected Column 类的实现类。
\end{itemize}

MBG配置中的其他几个标签基本上都是context的子标签,这些子标签(有严格的配置顺序,后面括号中的内容为这些标签可以配置的个数)包括以下几个: property(0+), plugin(0+), commentGenerator(0/1), jdbcConnection(1), javaTypeResolver(0/1), javaModelGenerator(1), sqlMapGenerator(0/1), javaClientGenerator(0/1), table(1+)。

\subsubsection*{property 标签}

假设数据库中有一个表,名为user info,中间有一个空格。在MySQL中可以使用反单引号“`”作为分隔符,例如`user info`,在SQL Server中则是[user info]。

通过分隔符可以将其中的内容作为一个整体的字符串进行处理,当SQL中有数据库关键字时,使用反单引号括住关键字,可以避免数据库产生错误。

property标签中包含了以下3个和分隔符相关的属性。
\begin{itemize}
    \item autoDelimitKeywords: 自动给关键字添加分隔符,MBG中维护了一个关键字列表,当数据库的字段或表与这些关键字一样时,MBG 会自动给这些字段或表添加分隔符。
    \item beginningDelimiter: 配置前置分隔符的属性。
    \item endingDelimiter: 配置后置分隔符的属性。
\end{itemize}

\begin{xml}
<property name="autoDelimitKeywords" value="true"/>
<property name="beginningDelimiter" value="`"/>
<property name="endingDelimiter" value="`"/>
\end{xml}

\subsubsection*{plugin 标签}

plugin 标签用来定义一个插件,用于扩展或修改通过 MBG 生成的代码。该插件将按在配置中配置的顺序执行。MBG插件使用的情况并不多。

以缓存插件 org.mybatis.generator.plugins.CachePlugin 为例。这个插件可以在生成的 SQL XML 映射文件中增加一个 cache 标签。只有当targetRuntime为MyBatis3时,该插件才有效。

可以这样进行配置(具体的插件属性不提及):

\begin{xml}
<plugin type="org.mybatis.generator.plugins.CachePlugin">
    <property name="cache_eviction" value="LRU"/>
    <property name="cache_size" value="1024"/>
</plugin>
\end{xml}

增加这个配置后,生成的 Mapper.xml 文件中会增加如下的缓存相关配置:

\begin{xml}
<cache eviction="LRU" size="1024">
...
</cache>
\end{xml}

\subsubsection*{commentGenerator 标签}

该标签用来配置如何生成注释信息,该标签有一个可选属性 type,可以指定用户的实现类,该类需要实现 org.mybatis.generator.api.CommentGenerator接口,而且必有一个默认空的构造方法。type属性接收默认的特殊值 DEFAULT,使用默认的实现类 DefaultCommentGenerator。

默认的实现类中提供了三个可选属性,需要通过property属性进行配置。
\begin{itemize}
    \item suppressAllComments: 阻止生成注释,默认为 false;
    \item suppressDate: 阻止生成的注释包含时间戳,默认为 false;
    \item addRemarkComments: 注释是否添加数据库表的备注信息,默认为 false;
\end{itemize}

一般情况下,由于 MBG 生成的注释信息没有任何价值,而且有时间戳的情况下每次生成的注释都不一样,使用版本控制的时候每次都会提交,因而一般情况下都会屏蔽注释信息,可以如下配置。

\begin{xml}
<commentGenerator>
    <property name="suppressDate" value="true"/>
    <property name="addRemarkComments" value="true"/>
</commentGenerator>
\end{xml}

自定义注释可以参考这篇文章: \url{https://blog.csdn.net/u011781521/article/details/78161201}

\subsubsection*{jdbcConnection 标签}

这个是用于连接数据库的标签,基础的四个属性不在说明。该标签还可以接受多个property子标签,这里配置的property属性都会添加到JDBC驱动的属性中。具体的值要看数据库的支持而定。

\subsubsection*{javaModelGenerator 标签}

该标签用来控制生成的实体类,根据context标签中配置的defaultModelType属性值的不同,一个表可能会对应生成多个不同的实体类。该标签只有两个必选属性:
\begin{itemize}
    \item targetPackage: 生成实体类存放的包名。一般就是放在该包下,实际还会受到其他配置的影响。
    \item targetProject:指定目标项目路径,可以使用相对路径或绝对路径。
\end{itemize}