\section{MyBatis 的注解使用方式}

MyBatis注解方式就是将SQL语句直接写在接口上。这种方式的优点是,对于需求比较简单的系统,效率较高。缺点是,当SQL有变化时都需要重新编译代码。更全面地使用注解开发应该学习 MyBatis-Plus。

\subsection{@Select 注解}

注解有多方便:

\begin{Java}
public interface RoleMapper {
    @Select("SELECT id, role_name, enabled, create_by, create_time FROM sys_role WHERE id = #{id}")
    SysRole selectById(Long id);
}
\end{Java}

然后就可以直接在测试类中运行了。

之前我们提到过 XML 配置有个问题: 如何实现字段映射? 同样有三种方式:
\begin{itemize}
    \item SQL语句使用别名;
    \item mybatis-plus 配置文件中开启 mapUnderscoreToCamelCase 配置。
    \item resultMap 方式。
\end{itemize}

前两种没什么好说的,和之前的操作一样,最后一种要用到 @Result 注解:

\begin{Java}
@Results(id = "roleResultMap", value = {
        @Result(property = "id", column = "id", id = true),
        @Result(property = "roleName", column = "role_name"),
        @Result(property = "enabled", column = "enabled"),
        @Result(property = "createBy", column = "create_by"),
        @Result(property = "createTime", column = "create_time")
})
@Select("SELECT id, role_name, enabled, create_by, create_time FROM sys_role WHERE id = #{id}")
SysRole selectById(Long id);
\end{Java}

有一个 id 属性是用来复用注解的。在 MyBatis3.3.1 及后续版本,可以使用 id 来服用 Results。

\begin{Java}
@ResultMap("roleResultMap")
@Select("select * from sys_role")
List<SysRole> selectAll();
\end{Java}

同时,如果 XML 配置了 resultMap,还可以根据对应的 id 获取。

\subsection{@Insert 注解}

Insert 本身是简单的,主要是返回值比较复杂。前面的会了,这里看一下例子就行了:

\begin{Java}
// 返回插入数据量
@Insert("INSERT INTO sys_role(id, role_name, enabled, create_by, create_time) VALUES (#{id}, #{roleName}, #{enabled}, #{createBy}, #{createTime jdbcType=TIMESTAMP})")
int insert(SysRole sysRole);

// 返回主键,方法 1
@Insert("INSERT INTO sys_role(role_name, enabled, create_by, create_time) VALUES #{roleName}, #{enabled}, #{createBy}, #{createTime jdbcType=TIMESTAMP})")
@Options(useGeneratedKeys = true, keyProperty = "id")
int insert2(SysRole sysRole);

// 返回主键,方法 2
@Insert("INSERT INTO sys_role(role_name, enabled, create_by, create_time) VALUES #{roleName}, #{enabled}, #{createBy}, #{createTime jdbcType=TIMESTAMP})")
@SelectKey(statement = "SELECT LAST_INSERT_ID()", keyProperty = "id", resultType = Long.class, before = false)
int insert3(SysRole sysRole);
\end{Java}

\subsection{@Update 和 @Delete 注解}

这两种更简单了:

\begin{Java}
@Update("UPDATE sys_role SET role_name = #{roleName}, enable = #{enabled}, create_by = {createBy}, create_time = #{createTime, jdbcType = TIMESTAMP} WHERE id = #{id}")
int updateById(SysRole sysRole);

@Delete("DELETE FROM sys_role WHERE id = #{id}")
int deleteById(Long id);
\end{Java}

\subsection{Provider 注解}

除了上面 4 中注解可以使用简单的 SQL 外,MyBatis 还提供了 4 中 Provider 注解,分别是 @SelectProvoder, @DeleteProvider, @UpdateProvider, @DeleteProvider。

下面来看看 @SelectProvider 的使用方法:

\begin{Java}
// 定义一个新 Provider 类
public class PrivilegeProvider {
    public String selectById(final Long id) {
        return new SQL(){
            {
                SELECT("id, privilege_name, privilege_url");
                FROM("sys_privilege");
                WHERE("id = #{id}");
            }
        }.toString();
    }
}
\end{Java}

\begin{Java}
// 使用 @SelectProvider
@SelectProvider(type = PrivilegeMapper.class, method = "selectById")
SysPrivilege selectById(Long id);
\end{Java}

Provider 提供了两个属性 type 和 method:
\begin{itemize}
    \item type: 包含 method 方法的类,这个类必须有空的构造方法,这个方法的值就是要执行的 SQL 语句。
    \item method: 对应方法的放回值必须是 String 类型。
\end{itemize}

前面的 method 语法比较复杂,但又代码提示,还可以直接写字符串,效果是一样的:

\begin{Java}
public String selectById2(final Long id) {
    return "SELECT id, privilege_name, privilege_url FROM sys_privilege WHERE id = #{id}";
}
\end{Java}

说白了 Provider 就是将 SQL 语句卸载单独的类中。

\newpage