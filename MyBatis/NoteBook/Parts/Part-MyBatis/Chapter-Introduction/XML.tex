\section{MyBatis 的 XML 使用方式}

\subsection{使用纯 XML 方式}

使用纯 XML 方式开发需要经历以下几个步骤:
\begin{itemize}
    \item 配置: 编写 mybatis-config.xml 配置文件。
    \item 映射: 编写 mapper 文件,统一管理 sql 语句。
    \item 编码: 在程序中获取并执行 SqlSession 语句。
\end{itemize}

\subsubsection{mybatis-config 配置文件}

使用 XML 形式进行配置,需要在 src/main/resources 下创建 mybatis-config.xml 文件\footnote{该配置来自MyBatis官网,包含了必要的配置信息,详细标签: \url{https://mybatis.net.cn/configuration.html}}:

\begin{xml}
<?xml version="1.0" encoding="UTF-8" ?>
<!DOCTYPE configuration
        PUBLIC "-//mybatis.org//DTD Config 3.0//EN"
        "http://mybatis.org/dtd/mybatis-3-config.dtd">
<configuration>
    <environments default="development">
        <environment id="development">
            <transactionManager type="JDBC"/>
            <dataSource type="POOLED">
                <property name="driver" value="${driver}"/>
                <property name="url" value="${url}"/>
                <property name="username" value="${username}"/>
                <property name="password" value="${password}"/>
            </dataSource>
        </environment>
    </environments>
    <mappers>
        <mapper resource="org/mybatis/example/BlogMapper.xml"/>
    </mappers>
</configuration>
\end{xml}

其中,<environments> 配置了数据库连接,<mappers> 中包含了一个 MyBatis 的 SQL 语句和映射配置文件,我们需要在对应的路径下创建映射文件,默认的,使用 Mapper 作为结尾(下文 mapper 文件就指的是 mybatis 的映射 xml 文件)。开发者可以根据需求修改或增添配置文件内容。

除此之外,我们还可以在 <configuration> 内添加下面这些元素:

\begin{xml}
<!-- 包别名,可以避免繁琐的完全限定名 -->
<typeAliases>
    <package name="learn.mybatis.simple.model"/>
</typeAliases>

<!-- 单独配置 sql 并引入 -->
<properties resource="sql.properties"/>

<!-- 使用日志 -->
<settings>
    <setting name="logImpl" value="LOG4J"/>
</settings>
\end{xml}

\subsubsection{mapper 映射文件}

在对应的映射文件中,我们可以添加自定义 sql 语句, 下面是一个例子:

\begin{xml}
<?xml version="1.0" encoding="UTF-8" ?>
<!DOCTYPE mapper PUBLIC "-//mybatis.org//DTD Mapper 3.0//EN"
        "http://mybatis.org/dtd/mybatis-3-mapper.dtd">
<mapper namespace="org.mybatis.example.BlogMapper">
    <select id="selectAll" resultType="Country">
        select id, country_name, country_code from country
    </select>
</mapper>
\end{xml}

其中,<mapper> 是根元素,后面紧跟当前 xml 的命名空间。<select> 代表这是一个查询操作,id 是唯一标识,resultType 对应我们自己写的实体类。相应的,我们还可以使用 update, delete 等其他元素,这里不一一介绍。

有了这些配置,我们就可以在代码中进行操作了:

\begin{Java}
// 通过 xml 获取 SessionFactory
Reader reader = Resources.getResourcesAsReader("mybatis-config.xml")
SqlSessionFactory sqlSessionFactory = new SqlSessionFactoryBuilder().build(reader);
reader.close();
// 通过 SessionFactory 使用已书写的 sql 语句
SqlSession sqlSession = sqlSessionFactory.openSession();
List<Country> countryList = sqlSession.selectList("selectAll");
sqlSession.close();
\end{Java}

上述方式是完全使用 xml 来配置 Mybatis, 这种方式已经不常用了,下文不再说明, MyBatis3 提供了接口来调用方法。

\subsection{Mapper 代理开发}

Mapper 代理开发是指结合映射文件与接口的开发方式,它不需要通过字面值(上文的 selectAll)这种硬编码方式执行,安全性更高,更加面向接口编程,可以看作是 MyBatis3 对 XML 方式的一种改进,是目前主流的开发方式之一。

如果映射文件过多,我们还可以在 <mappers> 中使用 <package> 元素包含对应包下的所有映射文件。下面两种写法效果相同:

使用这种方式开发需要经历以下几步:
\begin{itemize}
    \item 配置: 编写 mybatis-config.xml 配置文件,定义与映射文件名相同的 Mapper 接口,并且映射文件和接口文件必须在同一目录下。
    \item 映射: 设置映射文件的名称空间为 Mapper 接口的完全限定名。
    \item 接口: 在接口中定义方法,方法名就是映射文件中的 id, 保持参数类型和返回值类型一致。
    \item 编码: 通过 SqlSession 获取 Mapper 接口的代理对象并执行 sql 操作。
\end{itemize}

在配置过程中,使用代理开发的多个同目录的 mapper 映射可以改用 package 元素同一加载:

\begin{xml}
<mappers>
    <mapper resource="learn/mybatis/simple/mapper/UserMapper.xml"/>
    <mapper resource="learn/mybatis/simple/mapper/RoleMapper.xml"/>
    <mapper resource="learn/mybatis/simple/mapper/PrivilegeMapper.xml"/>
    <mapper resource="learn/mybatis/simple/mapper/UserRoleMapper.xml"/>
    <mapper resource="learn/mybatis/simple/mapper/RolePrivilegeMapper.xml"/>
</mappers>
<!-- 等效写法 -->
<mappers>
    <package name="learn/mybatis/simple/mapper"/>
</mappers>
\end{xml}

这种配置方法会先查找对应 java 包下的所有接口,并加载接口对应的 XML 映射文件。

同样,mapper 文件的 namespace 也需要修改:

\begin{xml}
<?xml version="1.0" encoding="UTF-8" ?>
<!DOCTYPE mapper PUBLIC "-//mybatis.org//DTD Mapper 3.0//EN"
        "http://mybatis.org/dtd/mybatis-3-mapper.dtd">
<mapper namespace="learn.mybatis.simple.mapper.PrivilegeMapper"/>
\end{xml}

\subsection{SELECT 用法}

\subsubsection{查询单个结果}

首先的,我们需要在 UserMapper.java 接口中声明 SELECT 方法,例如:
\begin{Java}
public interface UserMapper {
    SysUser selectById(long id);
}
\end{Java}

然后在对应的 UserMapper.xml 中添加下面代码:

\begin{xml}
<?xml version="1.0" encoding="UTF-8" ?>
<!DOCTYPE mapper PUBLIC "-//mybatis.org//DTD Mapper 3.0//EN"
        "http://mybatis.org/dtd/mybatis-3-mapper.dtd">
<mapper namespace="learn.mybatis.simple.mapper.UserMapper">
    <resultMap id="userMap" type="learn.mybatis.simple.mapper.SysUser">
        <id property="id" column="id"/>
        <result property="userName" column="user_name"/>
        <result property="userPassword" column="user_password"/>
        <result property="userEmail" column="user_email"/>
        <result property="userInfo" column="user_info"/>
        <result property="headImg" column="head_img" jdbcType="BLOB"/>
        <result property="createTime" column="create_time" jdbcType="TIMESTAMP"/>
    </resultMap>
    
    <select id="selectById" resultMap="userMap">
        select * from sys_user where id = #{id}
    </select>
</mapper>
\end{xml}

那么映射和接口文件/方法是怎么样绑定的呢?
\begin{itemize}
    \item 文件: 通过映射文件的 namespace 设置为接口的全限定名进行关联。如果不使用接口,则 namespace 可以随便写。
    \item 方法: 通过 <select> 标签的 id 属性值绑定。
\end{itemize}

映射文件中重要的标签和属性:
\begin{itemize}
    \item <select>: 映射查询语句使用的标签,id 属性与接口中的方法名对应。
    \item <resultMap>: 设置返回值的类型和映射关系。
    \item #{id}: MyBatis SQL中使用预编译参数的一种方式,大括号中的id是传入的参数名。
\end{itemize}

其中 <resultMap> 是最为重要的标签,它所包含的标签如下:
\begin{itemize}
    \item <constructor>: 配置使用构造方法注入结果,包含以下两个子标签。
    \begin{itemize}
        \item <idArg>: id 参数,标记结果作为 id,可以帮助提高性能。
        \item <arg>: 注入到构造方法的一个普通结果。
    \end{itemize}
    \item <id>: 一个 id 结果,标记结果作为 id, 可以帮助提高性能。
    \item <result>: 注入到 Java 对象属性的普通结果。
    \item <association>: 一个复杂的类型关联,许多结果将包成这种类型。
    \item <collection>: 复杂类型的集合。
    \item <discriminator>: 根据结果值来决定使用哪个结果映射。
    \item <case>: 基于某些值的结果映射。
\end{itemize}

更多标签的作用请查看官网文档,这里不再说明。

\subsubsection{查询多个结果}

加入我们查询的结果不是单个对象而是对象集,例如在 UserMapper.java 中添加如下方法:

\begin{Java}
List<SysUser> selectAll();
\end{Java}

对应的映射文件需要添加如下代码:

\begin{xml}
<select id="selectAll" resultType="learn.mybatis.simple.model.SysUser">
    SELECT id,
           user_name     userName,
           user_password userPassword,
           user_email    userEmail,
           user_info     userInfo,
           head_img      headImg,
           create_time   createTime
    FROM sys_user
</select>
\end{xml}

当返回结果大于1个时,必须使用 List<SysUser> 或 SysUser[] 作为返回类型,否则会抛 TooManyResultsException 异常。 

此外,在映射文件中我们使用了 resultType 返回结果类型,配合别名可以实现类型的自动映射,而不需要手动设置 <resultMap>

\paragraph*{名称映射规则} 如前文所述,MyBatis 有两种名称映射规则:
\begin{itemize}
    \item 通过 resultMap 手动设置结果类型。
    \item 通过 SQL 别名配合 resultType 实现类型自动映射。
\end{itemize}

注意,<result> 中的 property 要和对象中的属性名相同,而实际上,MyBatis 会将这两者都转换为大写进行判断。

我们知道,sql 语句是不区分大小写的,采用下划线命名,而 Java 属性一般采用小驼峰命名。MyBatis 十分贴心地为提供了一个全局属性 mapUnderscoreToCamelCase, 通过配置这个属性可以自动映射这两种命名。在 mybatis-config 文件中可以开启该属性:

\begin{xml}
<settings>
    <setting name="mapUnderscoreToCamelCase" value="true"/>
</settings>
\end{xml}

于是,我们就可以这样写了:
\begin{xml}
<select id="selectAll" resultType="learn.mybatis.simple.model.SysUser">
    SELECT id,
           user_name,
           user_password,
           user_email,
           user_info,
           head_img,
           create_time
    FROM sys_user
</select>
\end{xml}

当然,也可以直接 select *, 但出于性能考虑,最好不要这样做。


\newpage