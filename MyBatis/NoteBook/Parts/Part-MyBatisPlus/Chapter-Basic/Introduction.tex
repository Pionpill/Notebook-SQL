\section{MyBatis-Plus 简介}

\subsection{MyBatis-Plus 简介}

MyBatis-Plus (opens new window)(简称 MP)是一个 MyBatis (opens new window)的增强工具,在 MyBatis 的基础上只做增强不做改变,为简化开发、提高效率而生。MP 不是 MyBatis 的官方工具,而是国人开发的第三方工具。

说人话,MyBatis 要写很多 XML 映射文件,又需要定义很多接口,这使得开发过程要写的东西很多,MP 在其基础上,使用注解的方式简化了开发。

MP 对应的坐标(Spring Boot)如下:

\begin{xml}
<dependency>
    <groupId>com.baomidou</groupId>
    <artifactId>mybatis-plus-boot-starter</artifactId>
    <version>最新版本</version>
</dependency>
\end{xml}

只需要引入 MP 即可,不用再引入 MyBatis。

这部分默认读者已经会了 lombok,springboot 等出现的组件,下文默认基于 SpringBoot 开发。

\subsection{MyBatis-Plus 开发过程}

\subsubsection*{项目配置}

导入依赖后,需要在主程序同目录下创建两个包: pojo, mapper。

然后需要再主程序上加 @MapperScan 注解\footnote{@Mapper 注解用来替换映射文件中 <mapper> @MapperScan 则用于扫描 @Mapper。}。

\begin{Java}
@MapperScan("learn.example.mp.mapper")
@SpringBootApplication
public class MybatisPlusApplication {
    public static void main(String[] args) {
        SpringApplication.run(MybatisPlusApplication.class, args);
    }
}
\end{Java}


写入数据库配置:

\begin{xml}
#mysql
spring.datasource.driver-class-name=com.mysql.jdbc.Driver
spring.datasource.url=jdbc:mysql://localhost:3306/mp?useUnicode=true&characterEncoding=utf8
spring.datasource.username=root
spring.datasource.password=123456

#mybatis-plus
mybatis-plus.configuration.log-impl=org.apache.ibatis.logging.stdout.StdOutImpl
\end{xml}

然后创建数据,可以参考这篇文章: \url{https://blog.csdn.net/qq_38490457/article/details/108809902}

\subsubsection*{实体与接口}

接下来分别创建实体类: pojo.Employee 和接口 mapper.EmployeeMapper

\begin{Java}
@Data
@TableName("tbl_employee")
public class Employee {
    @TableId(value = "id", type= IdType.AUTO)
    private Integer id;

    @TableField("last_name")
    private String lastName;

    @TableField("email")
    private String email;

    @TableField("gender")
    private Integer gender;

    @TableField("age")
    private Integer age;
}
\end{Java}

\begin{Java}
@Repository
public interface EmployeeMapper extends BaseMapper<Employee> {
}
\end{Java}

到此,就全部完成了,常用的 SQL 操作已经被封装在了 BaseMapper 内。

相比 MyBatis,MP 做了这些:
\begin{itemize}
    \item 使用 @MapperScan 替换了映射文件,并自动读取了必要的配置。
    \item 通过注解和已封装好的接口,自动实现了基本的 CRUD 操作。
\end{itemize}

不得不说,MP 让 MyBatis 更像 JPA 了。