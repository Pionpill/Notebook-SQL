\documentclass{PionpillNote-book}
\usetikzlibrary {intersections,through,arrows.meta,graphs,shapes.misc,positioning,shapes.misc,positioning,calc}
\usetikzlibrary{animations}
\usetikzlibrary {shapes.geometric}
\usetikzlibrary {animations}
\usetikzlibrary {shapes.multipart}
\usetikzlibrary {positioning}
\usetikzlibrary {fit,shapes.geometric}
\usetikzlibrary {automata}
\usetikzlibrary {quotes}
\usetikzlibrary {matrix}
\usetikzlibrary {backgrounds}
\usetikzlibrary {scopes}
\usetikzlibrary {calc}
\usetikzlibrary {intersections}
\usetikzlibrary {svg.path}
\usetikzlibrary {decorations}
\usetikzlibrary {patterns}
\usetikzlibrary {decorations.pathmorphing}
\usetikzlibrary {shadows}
\usetikzlibrary {bending}

\title{MyBatis}
\author{
    Pionpill \footnote{笔名:北岸,电子邮件:673486387@qq.com,Github:\url{https://github.com/Pionpill}} \\
    本文档为作者归纳的 MyBatis 及 MyBatis Plus 笔记。\\
}

\date{\today}

\begin{document}

\pagestyle{plain}
\maketitle

\noindent\textbf{前言:}

笔者为软件工程系在校本科生,有计算机学科理论基础(操作系统,数据结构,计算机网络,编译原理等),本人在撰写此笔记时已有 Java, MySQL 基础。

本文的参考书籍如下:
\begin{itemize}
    \item MyBatis 部分: 《MyBatis 从入门到精通》 刘增辉 2017 第一版。
    \begin{itemize}
        \item 这是一本很好的带例子的入门书籍,建议看原文敲代码。
        \item 本人并没有写后几章内容,分别是插件开发和框架集成,有兴趣请看原书。
    \end{itemize}
    \item MyBatis-Plus 部分: 主要参考了官网和一些技术博客。
    \begin{itemize}
        \item 官网: \url{https://baomidou.com/}
        \item 博客: \url{https://blog.csdn.net/qq_38490457/article/details/108809902}
    \end{itemize}
\end{itemize}

本人的编写及开发环境如下:
\begin{itemize}
    \item Java: Java11
    \item OS: Windows11 
    \item MySQL: 8.0.3
    \item MyBatis: 3.5.11
\end{itemize}

\date{\today}
\newpage

\tableofcontents

\newpage

\setcounter{page}{1} 
\pagestyle{fancy}

\part{MyBatis}
\chapter{MyBatis 基础}
\import{Parts/Part-MyBatis/Chapter-Basic}{Introduction.tex}
\import{Parts/Part-MyBatis/Chapter-Basic}{XML.tex}
\import{Parts/Part-MyBatis/Chapter-Basic}{Annotation.tex}
\chapter{MyBatis 核心}
\import{Parts/Part-MyBatis/Chapter-Core}{DynamicSQL.tex}
\import{Parts/Part-MyBatis/Chapter-Core}{Generator.tex}
\import{Parts/Part-MyBatis/Chapter-Core}{Query.tex}
\import{Parts/Part-MyBatis/Chapter-Core}{Cache.tex}

\part{MyBatis-Plus}

\chapter{MyBatis-Plus 基础}
\import{Parts/Part-MyBatisPlus/Chapter-Basic}{Introduction.tex}



\end{document}